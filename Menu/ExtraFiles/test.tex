\documentclass[12pt, twoside]{article}
\usepackage[english]{babel}
\usepackage[utf8]{inputenc}
\usepackage[T1]{fontenc}
\usepackage[textwidth=15cm]{geometry}
\usepackage[x11names]{xcolor}
\usepackage{comment}
\usepackage{longtable}
\usepackage{transparent}
\usepackage{graphicx}
\usepackage{eso-pic}
\usepackage{epstopdf}
\usepackage{afterpage}

%\pagenumbering{}

%========================================================================================%
%                                                                                        %
%                					Function definitions								 %
%               Do not fiddle with this if you don't know what you are doing.            %
%                                                                                        %
%========================================================================================%

%======== Defines text size and style =======%
% used for descriptions 
\newcommand*\fontstyle[1]{%
  {\fontsize{15}{20}\sffamily{#1}}
}

% used for prices
\newcommand*\pricestyle[1]{%
  {\fontsize{14}{20}\normalfont#1}
}
  
% colorize text
\newcommand*\ColText[1]{\textcolor{Goldenrod3}{#1}}


%========= Defines two different environments ==========%

% environment for anything with one description and one price
\newenvironment{menuSection}[1]
  {
    \begin{longtable}{p{0cm}p{12cm}p{1.5cm}}
    {\fontsize{30}{34}\bf\selectfont\ColText{#1} }
    \\ \nopagebreak
  }
{\end{longtable}
hukh \pagebreak[3] \\ }

% environment for anything with one description and two prices
\newenvironment{whiskySection}[1]
  {
    \pagebreak[3]\noindent\begin{longtable}{p{0.3cm}p{10cm}p{1.5cm}p{1.5cm}}
    {\endfirsthead\fontsize{30}{34}\bf\selectfont\ColText{#1}} && \pricestyle{\underline{2 cl}} & \pricestyle{\underline{4 cl}}
    \\[0.4cm]\nopagebreak
  }  
{\end{longtable}}


%========== Defines list item appearance ===========%  

% The definition of the beer/cider/misc layout.
\newcommand*\beer[2]{
& \fontstyle{#1} & \pricestyle{#2 kr} \nopagebreak \\ [1.7ex]\nopagebreak[4]%
}

% The definition of the spirits (double priced) layout.
\newcommand*\spirit[3]{%
&\fontstyle{#1} & \pricestyle{#2 kr} & \pricestyle{#3 kr}\\ [1.7ex]\nopagebreak[4]%
}


% The command enables the transparent rydbergs image in the background of the first page
\newcommand{\FrontSeal}[1]{%
    \AddToShipoutPicture*{\AtPageCenter{%
    \makebox(0,0){\transparent{0.1}{\includegraphics%
	[width=0.9\paperwidth ]{#1}}}}}}
    
    
%%%%%%%%%%%%%%%%%%%%%%%%%%%%%%%%%%%%%%%%%%%%%%%%%%%%%%%%%%%%%%%%%%%%%%%%%%%%%%%%%%%%%%%%%% 
\begin{document}
hej
\newpage
\begin{titlepage}
\FrontSeal{Figures/JanneRydberg.jpg}
\begin{center}

\vspace*{8cm}

\includegraphics{Figures/rydbergtext.png}

\vspace*{4cm}
{\selectfont
{\small Updated \today}\\[2cm] % Date

{\Large We Always Serve Food!}\\
Refer to the blackboard on your right for details.
}

\vfill

\end{center}
\end{titlepage}

%========================================================================================%
%                                                                                        %
%        								Öl												 %
%                                                                                        %
%========================================================================================%

% This is the beer section. The group menuSection fixes a headline and sorts indentations
% out. 
% Items (beers) are added by the command:
% \beer{Beer name}{Price}
%
% if a desription is so long that it breaks the line you must add \\[-1ex] to the end
% ex: \beer{En öl som har ett jättelångt och jobbig namn... jätte jätte jättelångt}{pris}\\[-1ex]


\begin{menuSection}{Beer}

\beer{Bavaria}{15}
\beer{Bavaria Wit 0.0\% alcohol free}{15}
\beer{Budvar Premium}{20}
\beer{Budvar Dark 50 cl}{25}
\beer{Lindemans Apple 25 cl}{20}
\beer{Black Isle}{30}
\beer{Weinstephaner}{30}
\beer{Aecht Schlenkerla Rauchbier 50 cl}{35}
\end{menuSection}

%========================================================================================%
%                                                                                        %
%            							Cider											 %
%                                                                                        %
%========================================================================================%

% Beers and ciders are hadled similarly, so the cider entry command is still (counderintuitivly)
% \beer{cider name}{price}

\begin{menuSection}{Cider}
\beer{Angry Orchard}{25}
\beer{Green Goblin}{35}
\end{menuSection}


%========================================================================================%
%                                                                                        %
%                    					Wine        									 %
%                                                                                        %
%========================================================================================%

\begin{menuSection}{Wine}
\beer{Red wine Collino Rosso 1 glass}{25}
\beer{White wine Collino Bianco  1 glass}{25}
\end{menuSection}

var hamnar detta?

\newpage
\afterpage{\clearpage}
%========================================================================================%
%                                                                                        %
%                						Miscellaneous									 %
%                                                                                        %
%========================================================================================%


\begin{menuSection}{Miscellaneous}
\beer{Soda}{10}
\beer{Nut mix}{5}
\beer{Billy's pan pizza}{15}
\end{menuSection}


%========================================================================================%
%                                                                                        %
%                						Spirits											 %
%                                                                                        %
%========================================================================================%

\vspace{-1cm}
\begin{menuSection}{Spirits}\\& \emph{Prices for 4 cl}
\beer{House spirits}{25}
\\ \emptyline \\  & \noindent\emph{We appretiate if whisky and cocktail orders are postponed until after the food service is done.}&

\end{menuSection}
\\
%Det uppskattas om beställning av drinkar och whiskey uppskjutes till efter matserveringen

%========================================================================================%
%                                                                                        %
%            							Whisky											 %
%                                                                                        %
%========================================================================================%

% Tjenare Katja!
% 
% När du vill lägga till en sprit skriv \spirit{spritens namn}{pris för 2:a}{pris för 4:a} innanför \begin{whiskySection}{Whisky}   ...    \end{whiskySection}
% De sorteras inte av sig själva, så skriv in dem alfabetiskt och korrigera gärna priser, jag vet att vissa av dem är lite låga.
%




\begin{whiskySection}{Whisky}

\spirit{Ardbeg Corryvreckan}{20}{40}


\spirit{Auchentoshan 12y}{15}{30}


\spirit{Jura Superstition}{20}{35}

\spirit{Lagavulin 16y}{20}{35}

\spirit{Laphroaig Quarter Cask}{15}{30}

\spirit{Laphroaig 18y}{20}{40}

\spirit{Oban}{15}{30}

\spirit{Talisker 10y}{15}{25}

\end{whiskySection}




%========================================================================================%
%    									RUM												 %
%========================================================================================%
\begin{comment}

\begin{whiskySection}{Rums}

\spirit{Ron Zacapa}{35 kr}


\spirit{The Kraken}{25 kr}


\end{whiskySection}

\end{comment}


\end{document}
